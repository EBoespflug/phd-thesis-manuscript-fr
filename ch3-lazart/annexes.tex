       
\chapter{Documentation sur Lazart}

    Cette annexe présente différentes tables sur les paramètres et options de Lazart.

    \section{Paramètres d'une analyse dans l'API Python}
    \label{annexe:lz:analysis-args}

        La table \ref{tbl:lz:analysis-args} décrit les paramètres d'une analyse dans l'\gls{yaml} Python de Lazart.
        La colonne "Paramètre" indique le nom du paramètre. La colonne "Arg." indique si l'argument est positionnel (\texttt{argX}) ou par mot-clef (\texttt{kwargs}) et la colonne "Type" indique son type.
        La colonne "Défaut" correspond à la valeur par défaut de l'argument et "YAML" indique si cet argument a une correspondance dans le fichier de mutation \gls{yaml} ou s'il est réservé à l'\gls{api} Python.
            
        \begin{table}[htp]
        \scriptsize
        \centering
            \setlength\tabcolsep{3pt}
            \begin{tabular}{l|c|c|c|c|l}
            \multicolumn{1}{c|}{Paramètre} & Arg. & Type & Défaut & YAML & \multicolumn{1}{c}{Description} \\ \hline
            intput\_files & arg0 & \texttt{List[str]} & req. & non & Liste des fichier sources. \\ \hline
            attack\_model & arg1 & \texttt{str | dict} & req. & oui & \begin{tabular}[c]{@{}l@{}}Modèle d'attaquant, fourni soit comme le\\ chemin vers le fichier YAML, soit sous la\\ forme d'un dictionnaire décrivant les\\ modèles appliqués à chaque fonction.\end{tabular} \\ \hline
            path & kwarg & \texttt{str} & généré & non & \begin{tabular}[c]{@{}l@{}}Chemin du dossier dans lequel les fichiers\\ de l'analyse seront générés.\end{tabular} \\ \hline
            name & kwarg & \texttt{str} & \texttt{""} & non & Nom de l'analyse. \\ \hline
            flag & kwarg & \texttt{int} & \texttt{Atk} & non & \begin{tabular}[c]{@{}l@{}}Informations sur le type d'analyse, utilisé\\  pour des fonctions automatiques comme \\ texttt{execute} ou \texttt{report}.\end{tabular} \\ \hline
            max\_order & kwarg & \texttt{int} & 2 & non & Limite de faute pour l'analyse. \\ \hline
            rename\_bb & kwarg & \texttt{List[str]} & \texttt{"\_\_mut\_\_"} & oui & \begin{tabular}[c]{@{}l@{}}Description\footnote{\label{fn:tbl-lazart-analysis-params-mut} \texttt{"\_\_mut\_\_"} et \texttt{"\_\_mut\_\_"} correspondent respectivement à l'ensemble des fonctions sur lesquelles au moins un modèle est défini et à l'ensemble de toutes les fonctions.} des fonctions où appliquer de\\ renommage des blocs de base.\end{tabular} \\ \hline
            add\_trace & kwarg & \texttt{List[str]} & \texttt{"\_\_mut\_\_"} & oui & \begin{tabular}[c]{@{}l@{}}Description des fonctions où appliquer\\ la trace du chemin de l'exécution.\end{tabular} \\ \hline
            compiler & kwarg & \texttt{str} & \texttt{"clang"} & non & Nom de la commande du compilateur \\ \hline
            linker & kwarg & \texttt{str} & \texttt{"clang"} & non & Nom de la commande de l'éditeur des liens \\ \hline
            disassembler & kwarg & \texttt{str} & \texttt{"clang"} & non & Nom de la commande du désassembleur \\ \hline
            compiler\_args & \multicolumn{1}{l|}{kwarg} & \texttt{str} & \texttt{""} & non & \begin{tabular}[c]{@{}l@{}}Arguments supplémentaires pour l'appel de\\ l'éditeur des liens\end{tabular} \\ \hline
            linker\_args & kwarg & \texttt{str} & \texttt{""} & non & \begin{tabular}[c]{@{}l@{}}Arguments supplémentaires pour l'appel du\\ compilateur\end{tabular} \\ \hline
            dis\_args & kwarg & \texttt{str} & \texttt{""} & non & \begin{tabular}[c]{@{}l@{}}Arguments supplémentaires pour l'appel du\\ désassembleur\end{tabular} \\ \hline
            wolverine\_args & kwarg & \texttt{str} & \texttt{""} & non & \begin{tabular}[c]{@{}l@{}}Arguments supplémentaires pour l'appel de\\ Wolverine.\end{tabular} \\ \hline
            klee\_args & kwarg & \texttt{str} & \texttt{"emit-all-errors"} & non & Arguments pour l'appel de Klee. \\ \hline
            countermeasures & \multicolumn{1}{l|}{kwarg} & \multicolumn{1}{l|}{\texttt{List[dict]}} & \texttt{[]} & oui & \begin{tabular}[c]{@{}l@{}}Liste des contre-mesures automatiques à \\ appliquer sur le programme à analyser.\end{tabular}
            \end{tabular}
        \caption{Paramètres d'une analyse dans l'API Python.}
        \label{tbl:lz:analysis-args}
        \end{table}
                
    \section{Arguments des analyses}
    \label{annexe:lz:param-analysis}

        La table \ref{tbl:lz:param-analysis} décrit les arguments et options des différentes analyses (traitements) de Lazart.
        La colonne "Argument" indique le nom de l'argument et la colonne "Opt" indique si celui-ci est optionnel.
        "Type" correspond au type de l'argument et "Défaut" à sa valeur par défaute.
        Les colonnes "AA", "AAR", "HS", "DO" et "PL" indiquent si l'argument est disponible pour respectivement les analyses d'attaque, de redondance-équivalence, de points chauds, d'optimisation de détecteur et de placement.
    
        \begin{table}[htp]
        \footnotesize
        \centering
            \setlength\tabcolsep{3pt}
            \begin{tabular}{|l|l|l|l|l|l|l|l|l|}
            \hline
            Argument & Opt & Type & Défaut & AA & AAR & HS & DO & PL \\ \hline
            analysis & - & \texttt{(Trace) => bool} & \texttt{lambda t: t.satisfies()} & $\checkmark$ & $\checkmark$ & $\checkmark$ & $\checkmark$ & $\checkmark$ \\ \hline
            s\_fct & $\checkmark$ & \texttt{(Trace) => bool} & \texttt{lambda t: t.satisfies()} & $\checkmark$ & $\checkmark$ & $\checkmark$ & $\checkmark$ & - \\ \hline
            quiet & $\checkmark$ & \texttt{bool} & \texttt{false} & $\checkmark$ & $\checkmark$ & $\checkmark$ & $\checkmark$ & $\checkmark$ \\ \hline
            lazy & $\checkmark$ & \texttt{bool} & \texttt{false} & - & $\checkmark$ & - & - & - \\ \hline
            eq\_rule & $\checkmark$ & \texttt{(Trace, Trace) => bool} & Equivalence & - & $\checkmark$ & - & - & - \\ \hline
            red\_rule & $\checkmark$ & \texttt{(Trace, Trace) => bool} & Préfixe & - & $\checkmark$ & - & - & - \\ \hline
            do\_red & $\checkmark$ & \texttt{bool} & \texttt{true} & - & $\checkmark$ & - & - & - \\ \hline
            do\_eq & $\checkmark$ & \texttt{bool} & \texttt{true} & - & $\checkmark$ & - & - & - \\ \hline
            ccp\_list & $\checkmark$ & \texttt{[str]} & all & - & - & - & $\checkmark$ & - \\ \hline
            weight\_fct & $\checkmark$ & \texttt{([str]) => int} & \texttt{lambda dr: len(dr)} & - & - & - & $\checkmark$ & - \\ \hline
            \end{tabular} 
        \caption{Arguments des analyses}
        \label{tbl:lz:param-analysis}
        \end{table}
        
    \section{Defines macros de Lazart}
    \label{annexe:lz:defines}
    
        La table \ref{tbl:lz:defines} donne la liste des macros définies de Lazart lors de la compilation du programme, qui dépendent du type d'analyse et des options sélectionnées.
        Les colonnes "Macro" et "Type" correspondent respectivement à l'identifiant et au type de la macro.
        La colonne "Définition" précise si la macro est définie par Lazart en fonction de l'analyse ou précisée par l'utilisateur.
        
        \begin{table}[htp]
        \footnotesize
        \centering
            \setlength\tabcolsep{3pt}
            \begin{tabular}{|l|c|c|l|}
            \hline
            \multicolumn{1}{|c|}{Macro} & Type & Définition & \multicolumn{1}{c|}{Description} \\ \hline
            \texttt{LAZART} & define & lazart & Macro définie dans toute analyse avec Lazart. \\ \hline
            \texttt{\_LZ\_\_VERSION} & string & lazart & Détermine la version actuelle de Lazart core. \\ \hline
            \begin{tabular}[c]{@{}l@{}}\texttt{\_LZ\_\_VERSION\_MAJOR}\\ \texttt{\_LZ\_\_VERSION\_MINOR}\\ \texttt{\_LZ\_\_VERSION\_PATCH}\end{tabular} & int & lazart & \begin{tabular}[c]{@{}l@{}}Valeur entière correspondant au numéro de version \\ (major.minor.patch) de l'outil.\end{tabular} \\ \hline
            \texttt{\_LZ\_\_ATTACKS} & define & lazart & Macro définie si l'analyse est une analyse d'attaques. \\ \hline
            \texttt{\_LZ\_\_CPPO} & define & lazart & \begin{tabular}[c]{@{}l@{}}Macro définie si l'analyse est une analyse d'optimisation de\\ contre-mesures.\end{tabular} \\ \hline
            \texttt{\_LZ\_\_PLACEMENT} & define & lazart & \begin{tabular}[c]{@{}l@{}}Macro définie si l'analyse est une analyse de placement de\\ contre-mesures.\end{tabular} \\ \hline
            \texttt{\_LZ\_\_NO\_STD} & define & utilisateur & \begin{tabular}[c]{@{}l@{}}Si cette macro est définie, l'API C de Lazart est implémentée \\ sans utiliser la bibliothèque standard C.\end{tabular} \\ \hline
            \texttt{\_LZ\_\_MUT\_VALUE} & define & utilisateur & \begin{tabular}[c]{@{}l@{}}Active la récupération des valeurs injectées pour le modèle de\\ mutation de données.\end{tabular} \\ \hline
            \texttt{\_LZ\_\_CM\_USE\_EXIT} & \multicolumn{1}{l|}{define} & \multicolumn{1}{l|}{utilisateur} & \begin{tabular}[c]{@{}l@{}}Si cette macro est définie, les détecteurs stoppant sont implé-\\ mentés avec \texttt{exit} plutôt que \texttt{klee\_assume}.\end{tabular} \\ \hline
            \end{tabular}
        \caption{Macros définies dans Lazart}
        \label{tbl:lz:defines}
        \end{table}
        
    \section{Fonctions d'instrumentation de Lazart}
    \label{annexe:lz:instr-fct}
    
        La table \ref{tbl:lz:instr-fct} liste les différentes fonctions d'instrumentation supportées par Lazart.
        La colonne "Catégorie" précise le type de catégorie de la commande et "Commande" indique l'identifiant.
        La colonne "Args." correspond aux paramètres de la fonction ou de la macros et précise leurs types. La colonne "Étape" indique à quelle phase d'une analyse la fonction d'instrumentation est traitée par Lazart.
        Finalement la dernière colonne donne une description de la commande.
    
        \begin{sidewaystable}
            \scriptsize
            \setlength\tabcolsep{1.5pt}
            \begin{tabular}{|l|l|l|c|l|}
            \hline
            \multicolumn{1}{|c|}{Catégorie} & \multicolumn{1}{c|}{Commande} & \multicolumn{1}{c|}{Args.} & Étape & \multicolumn{1}{c|}{Description} \\ \hline
            \multirow{2}{*}{Utilitaire} & \texttt{\_LZ\_\_RENAME\_BB} & (str) & Pre-traitement & Renomme le bloc de base courant avec la chaîne de caractère spécifiée. \\ \cline{2-5} 
             & \texttt{\_LZ\_\_MAX\_DEPTH} & (int, expr) & DSE & \begin{tabular}[c]{@{}l@{}}Effectue une action lorsqu'un compteur local atteint une certaine valeur. Généralement utilisé\\ de manière à limiter l'exécution d'une boucle ou une récursion. Prend en argument la limite de\\ profondeur l'expression à exécuter.\end{tabular} \\ \hline
            \begin{tabular}[c]{@{}l@{}}Contrôles\\ des modèles\end{tabular} & \begin{tabular}[c]{@{}l@{}}\texttt{\_LZ\_\_DISABLE\_BB}\\ \texttt{\_LZ\_\_DISABLE\_FUNCTION}\end{tabular} & () & Mutation & Désactive toute mutation dans le bloc de base ou la fonction. \\ \hline
             & \begin{tabular}[c]{@{}l@{}}\texttt{\_LZ\_\_DISABLE\_MODEL}\\ \texttt{\_LZ\_\_ENABLE\_MODEL}\end{tabular} & (str) & Mutation & Désactive / réactive un modèle de faute nommé à partir de ce point de la mutation. \\ \hline
             & \begin{tabular}[c]{@{}l@{}}\texttt{\_LZ\_\_DISABLE\_MODELS}\\ \texttt{\_LZ\_\_ENABLE\_MODELS}\end{tabular} & () & Mutation & Désactive / réactive tous les modèles de fautes à partir de ce point de l'exécution symbolique. \\ \hline
             & \texttt{\_LZ\_\_RESET} & () & Mutation & Réinitialise tous les modèles activés et désactivés. \\ \hline
            \multirow{6}{*}{IP utilisateur} & \texttt{\_LZ\_\_mut\_ti} & (bool, str, str, str) & Mutation & \begin{tabular}[c]{@{}l@{}}Fonction de mutation pour l'inversion de test prenant en entrée la condition à fauter,\\ l'identifiant du point d'injection et les noms des deux blocs de base cibles.\end{tabular} \\ \cline{2-5} 
             & \begin{tabular}[c]{@{}l@{}}\texttt{\_LZ\_\_mut\_ti\_true}\\ \texttt{\_LZ\_\_mut\_ti\_false}\end{tabular} & (bool, str, str) & Mutation & Fonction de mutation pour l'inversion de test où seule l'une des branche peut être fautée. \\ \cline{2-5} 
             & \texttt{\_LZ\_\_mut\_dl\_fix\_iN} & (intN, intN, str) & Mutation & \begin{tabular}[c]{@{}l@{}}Fonctions de mutation pour l'injection de donnée fixe prenant en entrée la valeur originale,\\ la valeur fautée et l'identifiant du point d'injection.\end{tabular} \\ \cline{2-5} 
             & \texttt{\_LZ\_\_mut\_dl\_sym\_iN} & (intN, str) & Mutation & \begin{tabular}[c]{@{}l@{}}Fonctions de mutation pour l'injection de donnée arbitraire non-contrainte prenant en entrée\\ la valeur originale et l'identifiant du point d'injection.\end{tabular} \\ \cline{2-5} 
             & \begin{tabular}[c]{@{}l@{}}\texttt{\_LZ\_\_mut\_dl\_sym\_pred\_iN}\\ \texttt{\_LZ\_\_mut\_dl\_sym\_fct\_iN}\end{tabular} & \begin{tabular}[c]{@{}l@{}}(intN, str, \\     (intN) -> bool) \\ (intN, str,\\     (intN) -> intN)\end{tabular} & Mutation & \begin{tabular}[c]{@{}l@{}}Fonctions de mutation pour l'injection de donnée arbitraire contrainte prenant en entrée la\\ valeur originale, l'identifiant du point d'injection ainsi que la fonction de contrainte ou de\\ prédicat.\end{tabular} \\ \cline{2-5} 
             & \texttt{\_LZ\_\_mut\_jump} & (label, str) & Mutation & Macro pour le saut paramètré par un label de saut et un identifiant de point d'injection. \\ \hline
            \multirow{4}{*}{\begin{tabular}[c]{@{}l@{}}Objectif\\ d'attaque\end{tabular}} & \texttt{\_LZ\_\_ORACLE} & (bool) & DSE & \begin{tabular}[c]{@{}l@{}}Macro de définition de l'objectif d'attaque.\\ \texttt{Correspond à \texttt{klee\_assume}}.\end{tabular} \\ \cline{2-5} 
             & \texttt{\_LZ\_\_triggered} & () -> bool & DSE & \begin{tabular}[c]{@{}l@{}}Fonction permettant de déterminer si au moins une contre-mesure a été déclenchée sur\\ l'exécution courante.\end{tabular} \\ \cline{2-5} 
             & \begin{tabular}[c]{@{}l@{}}\texttt{\_LZ\_\_SYM}\\ \texttt{\_LZ\_\_SYM\_N}\end{tabular} & \begin{tabular}[c]{@{}l@{}}(ptr, size)\\ (ptr, size, str)\end{tabular} & DSE & \begin{tabular}[c]{@{}l@{}}Macros pour la définition de variables symbolique. Prend en paramètre la zone mémoire à\\ rendre symbolique et eventuellement un nom (utilisé par Klee).\\ \texttt{Correspond à \texttt{klee\_make\_symbolic}}.\end{tabular} \\ \cline{2-5} 
             & \texttt{\_LZ\_\_EVENT} & (str) & TP & \begin{tabular}[c]{@{}l@{}}Macro permettant d'ajouter un évènement utilisateur dans la trace, pouvant ensuite être trié\\ dans le script Python. Prend en paramètre une chaîne de caractère de description de l'évènement.\end{tabular} \\ \hline
            \multirow{2}{*}{Contre-mesures} & \texttt{\_LZ\_\_CM} & (str) & TP & \begin{tabular}[c]{@{}l@{}}Macro de déclenchement d'un détecteur, utilise la version bloquante ou non-bloquante en\\ fonction de l'analyse. Prend en paramètre l'identifiant du détecteur.\end{tabular} \\ \cline{2-5} 
             & \begin{tabular}[c]{@{}l@{}}\texttt{\_LZ\_\_trigger\_detector}\\ \texttt{\_LZ\_\_trigger\_detector\_stop}\end{tabular} & (str) & TP & Fonctions de déclenchement de détecteur. Prend en paramètre l'identifiant du détecteur. \\ \hline
            \end{tabular}
        \caption{Fonctions d'instrumentation dans Lazart}
        \label{tbl:lz:instr-fct}
        \end{sidewaystable}