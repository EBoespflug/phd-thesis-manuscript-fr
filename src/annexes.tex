       
\chapter{Documentation sur Lazart}

    Cette annexe présente différentes tables sur les paramètres et options de Lazart.

    \section{Paramètres d'une analyse dans l'API Python}
    \label{annexe:lz:analysis-args}

        La table \ref{tbl:lz:analysis-args} décrit les paramètres d'une analyse dans l'\gls{yaml} Python de Lazart.
        La colonne "Paramètre" indique le nom du paramètre. La colonne "Arg." indique si l'argument est positionnel (\texttt{argX}) ou par mot-clef (\texttt{kwargs}) et la colonne "Type" indique son type.
        La colonne "Défaut" correspond à la valeur par défaut de l'argument et "YAML" indique si cet argument a une correspondance dans le fichier de mutation \gls{yaml} ou s'il est réservé à l'\gls{api} Python.
            
        \begin{table}[htp]
        \scriptsize
        \centering
            \setlength\tabcolsep{3pt}
            \begin{tabular}{l|c|c|c|c|l}
            \multicolumn{1}{c|}{Paramètre} & Arg. & Type & Défaut & YAML & \multicolumn{1}{c}{Description} \\ \hline
            intput\_files & arg0 & \texttt{List[str]} & req. & non & Liste des fichier sources. \\ \hline
            attack\_model & arg1 & \texttt{str | dict} & req. & oui & \begin{tabular}[c]{@{}l@{}}Modèle d'attaquant, fourni soit comme le\\ chemin vers le fichier YAML, soit sous la\\ forme d'un dictionnaire décrivant les\\ modèles appliqués à chaque fonction.\end{tabular} \\ \hline
            path & kwarg & \texttt{str} & généré & non & \begin{tabular}[c]{@{}l@{}}Chemin du dossier dans lequel les fichiers\\ de l'analyse seront générés.\end{tabular} \\ \hline
            name & kwarg & \texttt{str} & \texttt{""} & non & Nom de l'analyse. \\ \hline
            flag & kwarg & \texttt{int} & \texttt{Atk} & non & \begin{tabular}[c]{@{}l@{}}Informations sur le type d'analyse, utilisé\\  pour des fonctions automatiques comme \\ texttt{execute} ou \texttt{report}.\end{tabular} \\ \hline
            max\_order & kwarg & \texttt{int} & 2 & non & Limite de faute pour l'analyse. \\ \hline
            rename\_bb & kwarg & \texttt{List[str]} & \texttt{"\_\_mut\_\_"} & oui & \begin{tabular}[c]{@{}l@{}}Description\footnote{\label{fn:tbl-lazart-analysis-params-mut} \texttt{"\_\_mut\_\_"} et \texttt{"\_\_mut\_\_"} correspondent respectivement à l'ensemble des fonctions sur lesquelles au moins un modèle est défini et à l'ensemble de toutes les fonctions.} des fonctions où appliquer de\\ renommage des blocs de base.\end{tabular} \\ \hline
            add\_trace & kwarg & \texttt{List[str]} & \texttt{"\_\_mut\_\_"} & oui & \begin{tabular}[c]{@{}l@{}}Description des fonctions où appliquer\\ la trace du chemin de l'exécution.\end{tabular} \\ \hline
            compiler & kwarg & \texttt{str} & \texttt{"clang"} & non & Nom de la commande du compilateur \\ \hline
            linker & kwarg & \texttt{str} & \texttt{"clang"} & non & Nom de la commande de l'éditeur des liens \\ \hline
            disassembler & kwarg & \texttt{str} & \texttt{"clang"} & non & Nom de la commande du désassembleur \\ \hline
            compiler\_args & \multicolumn{1}{l|}{kwarg} & \texttt{str} & \texttt{""} & non & \begin{tabular}[c]{@{}l@{}}Arguments supplémentaires pour l'appel de\\ l'éditeur des liens\end{tabular} \\ \hline
            linker\_args & kwarg & \texttt{str} & \texttt{""} & non & \begin{tabular}[c]{@{}l@{}}Arguments supplémentaires pour l'appel du\\ compilateur\end{tabular} \\ \hline
            dis\_args & kwarg & \texttt{str} & \texttt{""} & non & \begin{tabular}[c]{@{}l@{}}Arguments supplémentaires pour l'appel du\\ désassembleur\end{tabular} \\ \hline
            wolverine\_args & kwarg & \texttt{str} & \texttt{""} & non & \begin{tabular}[c]{@{}l@{}}Arguments supplémentaires pour l'appel de\\ Wolverine.\end{tabular} \\ \hline
            klee\_args & kwarg & \texttt{str} & \texttt{"emit-all-errors"} & non & Arguments pour l'appel de Klee. \\ \hline
            countermeasures & \multicolumn{1}{l|}{kwarg} & \multicolumn{1}{l|}{\texttt{List[dict]}} & \texttt{[]} & oui & \begin{tabular}[c]{@{}l@{}}Liste des contre-mesures automatiques à \\ appliquer sur le programme à analyser.\end{tabular}
            \end{tabular}
        \caption{Paramètres d'une analyse dans l'API Python.}
        \label{tbl:lz:analysis-args}
        \end{table}
                
    \section{Arguments des analyses}
    \label{annexe:lz:param-analysis}

        La table \ref{tbl:lz:param-analysis} décrit les arguments et options des différentes analyses (traitements) de Lazart.
        La colonne "Argument" indique le nom de l'argument et la colonne "Opt" indique si celui-ci est optionnel.
        "Type" correspond au type de l'argument et "Défaut" à sa valeur par défaute.
        Les colonnes "AA", "AAR", "HS", "DO" et "PL" indiquent si l'argument est disponible pour respectivement les analyses d'attaque, de redondance-équivalence, de points chauds, d'optimisation de détecteur et de placement.
    
        \begin{table}[htp]
        \footnotesize
        \centering
            \setlength\tabcolsep{3pt}
            \begin{tabular}{|l|l|l|l|l|l|l|l|l|}
            \hline
            Argument & Opt & Type & Défaut & AA & AAR & HS & DO & PL \\ \hline
            analysis & - & \texttt{(Trace) => bool} & \texttt{lambda t: t.satisfies()} & $\checkmark$ & $\checkmark$ & $\checkmark$ & $\checkmark$ & $\checkmark$ \\ \hline
            s\_fct & $\checkmark$ & \texttt{(Trace) => bool} & \texttt{lambda t: t.satisfies()} & $\checkmark$ & $\checkmark$ & $\checkmark$ & $\checkmark$ & - \\ \hline
            quiet & $\checkmark$ & \texttt{bool} & \texttt{false} & $\checkmark$ & $\checkmark$ & $\checkmark$ & $\checkmark$ & $\checkmark$ \\ \hline
            lazy & $\checkmark$ & \texttt{bool} & \texttt{false} & - & $\checkmark$ & - & - & - \\ \hline
            eq\_rule & $\checkmark$ & \texttt{(Trace, Trace) => bool} & Equivalence & - & $\checkmark$ & - & - & - \\ \hline
            red\_rule & $\checkmark$ & \texttt{(Trace, Trace) => bool} & Préfixe & - & $\checkmark$ & - & - & - \\ \hline
            do\_red & $\checkmark$ & \texttt{bool} & \texttt{true} & - & $\checkmark$ & - & - & - \\ \hline
            do\_eq & $\checkmark$ & \texttt{bool} & \texttt{true} & - & $\checkmark$ & - & - & - \\ \hline
            ccp\_list & $\checkmark$ & \texttt{[str]} & all & - & - & - & $\checkmark$ & - \\ \hline
            weight\_fct & $\checkmark$ & \texttt{([str]) => int} & \texttt{lambda dr: len(dr)} & - & - & - & $\checkmark$ & - \\ \hline
            \end{tabular} 
        \caption{Arguments des analyses}
        \label{tbl:lz:param-analysis}
        \end{table}
        
    \section{Defines macros de Lazart}
    \label{annexe:lz:defines}
    
        La table \ref{tbl:lz:defines} donne la liste des macros définies de Lazart lors de la compilation du programme, qui dépendent du type d'analyse et des options sélectionnées.
        Les colonnes "Macro" et "Type" correspondent respectivement à l'identifiant et au type de la macro.
        La colonne "Définition" précise si la macro est définie par Lazart en fonction de l'analyse ou précisée par l'utilisateur.
        
        \begin{table}[htp]
        \footnotesize
        \centering
            \setlength\tabcolsep{3pt}
            \begin{tabular}{|l|c|c|l|}
            \hline
            \multicolumn{1}{|c|}{Macro} & Type & Définition & \multicolumn{1}{c|}{Description} \\ \hline
            \texttt{LAZART} & define & lazart & Macro définie dans toute analyse avec Lazart. \\ \hline
            \texttt{\_LZ\_\_VERSION} & string & lazart & Détermine la version actuelle de Lazart core. \\ \hline
            \begin{tabular}[c]{@{}l@{}}\texttt{\_LZ\_\_VERSION\_MAJOR}\\ \texttt{\_LZ\_\_VERSION\_MINOR}\\ \texttt{\_LZ\_\_VERSION\_PATCH}\end{tabular} & int & lazart & \begin{tabular}[c]{@{}l@{}}Valeur entière correspondant au numéro de version \\ (major.minor.patch) de l'outil.\end{tabular} \\ \hline
            \texttt{\_LZ\_\_ATTACKS} & define & lazart & Macro définie si l'analyse est une analyse d'attaques. \\ \hline
            \texttt{\_LZ\_\_CPPO} & define & lazart & \begin{tabular}[c]{@{}l@{}}Macro définie si l'analyse est une analyse d'optimisation de\\ contre-mesures.\end{tabular} \\ \hline
            \texttt{\_LZ\_\_PLACEMENT} & define & lazart & \begin{tabular}[c]{@{}l@{}}Macro définie si l'analyse est une analyse de placement de\\ contre-mesures.\end{tabular} \\ \hline
            \texttt{\_LZ\_\_NO\_STD} & define & utilisateur & \begin{tabular}[c]{@{}l@{}}Si cette macro est définie, l'API C de Lazart est implémentée \\ sans utiliser la bibliothèque standard C.\end{tabular} \\ \hline
            \texttt{\_LZ\_\_MUT\_VALUE} & define & utilisateur & \begin{tabular}[c]{@{}l@{}}Active la récupération des valeurs injectées pour le modèle de\\ mutation de données.\end{tabular} \\ \hline
            \texttt{\_LZ\_\_CM\_USE\_EXIT} & \multicolumn{1}{l|}{define} & \multicolumn{1}{l|}{utilisateur} & \begin{tabular}[c]{@{}l@{}}Si cette macro est définie, les détecteurs stoppant sont implé-\\ mentés avec \texttt{exit} plutôt que \texttt{klee\_assume}.\end{tabular} \\ \hline
            \end{tabular}
        \caption{Macros définies dans Lazart}
        \label{tbl:lz:defines}
        \end{table}
        
    \section{Fonctions d'instrumentation de Lazart}
    \label{annexe:lz:instr-fct}
    
        La table \ref{tbl:lz:instr-fct} liste les différentes fonctions d'instrumentation supportées par Lazart.
        La colonne "Catégorie" précise le type de catégorie de la commande et "Commande" indique l'identifiant.
        La colonne "Args." correspond aux paramètres de la fonction ou de la macros et précise leurs types. La colonne "Étape" indique à quelle phase d'une analyse la fonction d'instrumentation est traitée par Lazart.
        Finalement la dernière colonne donne une description de la commande.
    
        \begin{sidewaystable}
            \scriptsize
            \setlength\tabcolsep{1.5pt}
            \begin{tabular}{|l|l|l|c|l|}
            \hline
            \multicolumn{1}{|c|}{Catégorie} & \multicolumn{1}{c|}{Commande} & \multicolumn{1}{c|}{Args.} & Étape & \multicolumn{1}{c|}{Description} \\ \hline
            \multirow{2}{*}{Utilitaire} & \texttt{\_LZ\_\_RENAME\_BB} & (str) & Pre-traitement & Renomme le bloc de base courant avec la chaîne de caractère spécifiée. \\ \cline{2-5} 
             & \texttt{\_LZ\_\_MAX\_DEPTH} & (int, expr) & DSE & \begin{tabular}[c]{@{}l@{}}Effectue une action lorsqu'un compteur local atteint une certaine valeur. Généralement utilisé\\ de manière à limiter l'exécution d'une boucle ou une récursion. Prend en argument la limite de\\ profondeur l'expression à exécuter.\end{tabular} \\ \hline
            \begin{tabular}[c]{@{}l@{}}Contrôles\\ des modèles\end{tabular} & \begin{tabular}[c]{@{}l@{}}\texttt{\_LZ\_\_DISABLE\_BB}\\ \texttt{\_LZ\_\_DISABLE\_FUNCTION}\end{tabular} & () & Mutation & Désactive toute mutation dans le bloc de base ou la fonction. \\ \hline
             & \begin{tabular}[c]{@{}l@{}}\texttt{\_LZ\_\_DISABLE\_MODEL}\\ \texttt{\_LZ\_\_ENABLE\_MODEL}\end{tabular} & (str) & Mutation & Désactive / réactive un modèle de faute nommé à partir de ce point de la mutation. \\ \hline
             & \begin{tabular}[c]{@{}l@{}}\texttt{\_LZ\_\_DISABLE\_MODELS}\\ \texttt{\_LZ\_\_ENABLE\_MODELS}\end{tabular} & () & Mutation & Désactive / réactive tous les modèles de fautes à partir de ce point de l'exécution symbolique. \\ \hline
             & \texttt{\_LZ\_\_RESET} & () & Mutation & Réinitialise tous les modèles activés et désactivés. \\ \hline
            \multirow{6}{*}{IP utilisateur} & \texttt{\_LZ\_\_mut\_ti} & (bool, str, str, str) & Mutation & \begin{tabular}[c]{@{}l@{}}Fonction de mutation pour l'inversion de test prenant en entrée la condition à fauter,\\ l'identifiant du point d'injection et les noms des deux blocs de base cibles.\end{tabular} \\ \cline{2-5} 
             & \begin{tabular}[c]{@{}l@{}}\texttt{\_LZ\_\_mut\_ti\_true}\\ \texttt{\_LZ\_\_mut\_ti\_false}\end{tabular} & (bool, str, str) & Mutation & Fonction de mutation pour l'inversion de test où seule l'une des branche peut être fautée. \\ \cline{2-5} 
             & \texttt{\_LZ\_\_mut\_dl\_fix\_iN} & (intN, intN, str) & Mutation & \begin{tabular}[c]{@{}l@{}}Fonctions de mutation pour l'injection de donnée fixe prenant en entrée la valeur originale,\\ la valeur fautée et l'identifiant du point d'injection.\end{tabular} \\ \cline{2-5} 
             & \texttt{\_LZ\_\_mut\_dl\_sym\_iN} & (intN, str) & Mutation & \begin{tabular}[c]{@{}l@{}}Fonctions de mutation pour l'injection de donnée arbitraire non-contrainte prenant en entrée\\ la valeur originale et l'identifiant du point d'injection.\end{tabular} \\ \cline{2-5} 
             & \begin{tabular}[c]{@{}l@{}}\texttt{\_LZ\_\_mut\_dl\_sym\_pred\_iN}\\ \texttt{\_LZ\_\_mut\_dl\_sym\_fct\_iN}\end{tabular} & \begin{tabular}[c]{@{}l@{}}(intN, str, \\     (intN) -> bool) \\ (intN, str,\\     (intN) -> intN)\end{tabular} & Mutation & \begin{tabular}[c]{@{}l@{}}Fonctions de mutation pour l'injection de donnée arbitraire contrainte prenant en entrée la\\ valeur originale, l'identifiant du point d'injection ainsi que la fonction de contrainte ou de\\ prédicat.\end{tabular} \\ \cline{2-5} 
             & \texttt{\_LZ\_\_mut\_jump} & (label, str) & Mutation & Macro pour le saut paramètré par un label de saut et un identifiant de point d'injection. \\ \hline
            \multirow{4}{*}{\begin{tabular}[c]{@{}l@{}}Objectif\\ d'attaque\end{tabular}} & \texttt{\_LZ\_\_ORACLE} & (bool) & DSE & \begin{tabular}[c]{@{}l@{}}Macro de définition de l'objectif d'attaque.\\ \texttt{Correspond à \texttt{klee\_assume}}.\end{tabular} \\ \cline{2-5} 
             & \texttt{\_LZ\_\_triggered} & () -> bool & DSE & \begin{tabular}[c]{@{}l@{}}Fonction permettant de déterminer si au moins une contre-mesure a été déclenchée sur\\ l'exécution courante.\end{tabular} \\ \cline{2-5} 
             & \begin{tabular}[c]{@{}l@{}}\texttt{\_LZ\_\_SYM}\\ \texttt{\_LZ\_\_SYM\_N}\end{tabular} & \begin{tabular}[c]{@{}l@{}}(ptr, size)\\ (ptr, size, str)\end{tabular} & DSE & \begin{tabular}[c]{@{}l@{}}Macros pour la définition de variables symbolique. Prend en paramètre la zone mémoire à\\ rendre symbolique et eventuellement un nom (utilisé par Klee).\\ \texttt{Correspond à \texttt{klee\_make\_symbolic}}.\end{tabular} \\ \cline{2-5} 
             & \texttt{\_LZ\_\_EVENT} & (str) & TP & \begin{tabular}[c]{@{}l@{}}Macro permettant d'ajouter un évènement utilisateur dans la trace, pouvant ensuite être trié\\ dans le script Python. Prend en paramètre une chaîne de caractère de description de l'évènement.\end{tabular} \\ \hline
            \multirow{2}{*}{Contre-mesures} & \texttt{\_LZ\_\_CM} & (str) & TP & \begin{tabular}[c]{@{}l@{}}Macro de déclenchement d'un détecteur, utilise la version bloquante ou non-bloquante en\\ fonction de l'analyse. Prend en paramètre l'identifiant du détecteur.\end{tabular} \\ \cline{2-5} 
             & \begin{tabular}[c]{@{}l@{}}\texttt{\_LZ\_\_trigger\_detector}\\ \texttt{\_LZ\_\_trigger\_detector\_stop}\end{tabular} & (str) & TP & Fonctions de déclenchement de détecteur. Prend en paramètre l'identifiant du détecteur. \\ \hline
            \end{tabular}
        \caption{Fonctions d'instrumentation dans Lazart}
        \label{tbl:lz:instr-fct}
        \end{sidewaystable}

\chapter{Programmes d'expérimentation}

    Cette annexe présente plusieurs programmes utilisés dans différentes expérimentation de ce manuscrit.
    Il s'agit de donner le code des programmes ainsi que les objectifs d'attaque qui sont étudiés.
    Le code des programmes a été légèrement modifié de manière à rendre les codes auto-suffisant et certaines macros, permettant la compilation conditionnelle de différentes versions protégées du programme, ont été retirée par soucis de simplicité.

    La section \ref{annexe:prgm:vp2b} présente le programme \texttt{verify\_pin\_2b}.
    La section \ref{annexe:prgm:memcmps3} décrit le programme \texttt{memcmps3}. 
    La section \ref{annexe:prgm:fu2} présente le programme \texttt{firmware\_updater\_2}.

    \section{Programme \texttt{verify\_pin\_2b}}
    \label{annexe:prgm:vp2b}
    
        Le programme \textit{verify\_pin\_2b} (listing \ref{lst:vp2b}) est une version du programme \texttt{verify\_pin} comprenant les booléens endurcis et la boucle en temps constant. Contrairement à la version \texttt{verify\_pin\_2} de la collection \gls{fissc}, il n'y a pas de contre-mesure visant à vérifier le compteur de boucle.
    
        Ce programme a notamment été utilisé dans le chapitre \ref{chpt:placement} en tant que programme d'exemples pour les expérimentations (voir section \ref{sec:placement-exps}).
        Il est analysé avec le modèles \gls{TI} (ou \gls{BI}).

\lstset{caption={Programme d'analyse pour \texttt{verify\_pin\_2b}},label=lst:vp2b,language=C}
\begin{lstlisting}
#include "lazart.h"

typedef int sec_bool_;
typedef enum { sec_true = 0xAA,
    sec_false = 0x55 } sec_bool_t;

#define BOOL sec_bool_t
#define TRUE sec_true
#define FALSE sec_false

#define PIN_SIZE 4
#define TRY_COUNT 3

uint8_t try_counter;
uint8_t card_pin[PIN_SIZE];

BOOL compare(uint8_t* a1, uint8_t* a2, size_t size)
{
    int i;
    bool diff = FALSE;
    for (i = 0; i < size; i++) { // OT
        if (a1[i] != a2[i]) { // 1T
            diff = TRUE;
        }
        // 1F
    } // OF
    return !diff;
}


BOOL verify_pin(uint8_t* user_pin)
{
    if (try_counter > 0) { // 2T
        if (compare(card_pin, user_pin, PIN_SIZE) == TRUE) { // 3F
            try_counter = TRY_COUNT;
            return TRUE;
        } else { // 3F
            try_counter--;
            return FALSE;
        }
    }
    // 2F

    return FALSE;
}


void init(uint8_t* user_pin, uint8_t* card_pin)
{
    try_counter = TRY_COUNT;
    klee_make_symbolic(user_pin, PIN_SIZE, "user_pin");
    klee_make_symbolic(card_pin, PIN_SIZE, "card_pin");

    int equal = (card_pin[0] == user_pin[0]) & (card_pin[1] == user_pin[1]) & (card_pin[2] == user_pin[2]) & (card_pin[3] == user_pin[3]);

    klee_assume(!equal);
}

bool oracle(BOOL ret) { 
    return ret == TRUE & !_LZ__triggered();
}

int main()
{
    uint8_t user_pin[PIN_SIZE];

    init(user_pin, card_pin);

    bool ret = verify_pin(user_pin);
    _LZ__ORACLE(oracle(ret));

    return 0;
}
\end{lstlisting}

    \section{Programme \texttt{memcmps3}}
    \label{annexe:prgm:memcmps3}
        Le programme \texttt{memcmps3} (listing \ref{lst:memcmps3}) correspond à une versions protégée de la fonction standard \texttt{memcmp}, utilisant des masques.
        La version 3 contient 4 appels à la fonction standard, contrairement à la version 2 (voir section \ref{sec:lz:memcmps}) qui n'en contient que 3.
    
        L'objectif d'attaque étudié correspond à retourner \texttt{TRUE}, avec des tableaux d'entrées différents.
        Le programme peut être analysé avec le modèle \gls{TI} ou en mutation de donnée sur les variables \texttt{len} et \texttt{result}.  

\lstset{caption={Programme d'analyse pour \texttt{memcmps3}},label=lst:memcmps3,language=C}
\begin{lstlisting}  
#include "lazart.h"
#include <stdio.h>

#define TRUE 0x1234
#define FALSE 0x5678
#define MASK 0xABCD
#define MASK2 0xF4F4

int memcmps3(char* a, char* b, int len)
{
    int result = FALSE;
    if (!memcmp(a, b, len)) {
        result ^= FALSE ^ MASK; // result = MASK
        if (!memcmp(a, b, len)) {
            result ^= MASK2; // result = MASK ^ MASK2
            if (!memcmp(a, b, len)) {
                result ^= TRUE ^ MASK; // result = MASK2 ^ TRUE
                if (!memcmp(a, b, len))
                    result ^= MASK2; // result = TRUE
            }
        }
    }

    return result;
}

void initialize_sym(char s1[SIZE], char s2[SIZE])
{
    klee_make_symbolic(s1, SIZE * sizeof(char), "s1");
    klee_make_symbolic(s2, SIZE * sizeof(char), "s2");
    int equal = 0;
    for (unsigned int i = 0; i < SIZE; i++) {
        equal += s1[i] == s2[i];
    }
    klee_assume(equal != SIZE);
}

int main()
{
    char s1[SIZE], s2[SIZE];
    initialize_sym(s1, s2);

    int res;
    res = memcmps3(s1, s2, SIZE);
    
    _LZ__ORACLE(oracle(res));
    return 0;
}
\end{lstlisting} 

    \section{Programme \texttt{firmware\_updater} 2}
    \label{annexe:prgm:fu2}
    
        Les programmes \texttt{firmware\_updater} simulent le processus de chargement d'un micro-programme depuis le réseau et la mise à jour du code local.
        Le listing \ref{lst:firmware-updater-2} correspond à la version 2 de ce programme, qui inclut une checksum permettant de vérifier l'intégrité à différents points du programme.
    
        L'objectif d'attaque étudié consiste à réussir à corrompre le micro-programme, que ce soit en écrivant à la mauvaise adresse, ou en chargeant des données corrompues.
        L'intégrité du micro-programme effectivement chargé en mémoire est vérifiée dans l'objectif d'attaque.
    
\lstset{caption={Programme d'analyse pour \texttt{firmware\_updater\_2}},label=lst:firmware-updater-2}
\begin{lstlisting} 
#include "lazart.h"
#include "stdbool.h"
#include "stdint.h"
#include <stdio.h>
#include <stdlib.h>


#define PAGE_NUMBER 4 // number of pages in a firmware
#define SIZEP 3 // number of bytes in a page
#define LOAD_ADDRESS 0x0F // default load address value for the new firmware
#define BAD_ADDRESS 0x0 // represents any corrupted load address ...

#define NW_CANARY() 3
#define NW_SIZE_BASE() (PAGE_NUMBER * SIZEP)
#ifndef GLOBAL_CHECKSUM
#define NW_SIZE_GCS() 0
#else
#define NW_SIZE_GCS() 1
#endif
#ifndef NO_CHECKSUM
#define NW_SIZE_CS() (PAGE_NUMBER) 
#else
#define NW_SIZE_CS() 0
#endif
#define NW_SIZE() (NW_SIZE_BASE() + NW_SIZE_CS() + NW_SIZE_GCS() + NW_CANARY())

#include <stdio.h>
#include <stdint.h>

typedef uint8_t byte_t;

typedef struct {
    byte_t t[SIZEP];
    byte_t checksum;
} page_t;


typedef struct {
    page_t pages[PAGE_NUMBER];
} firmware_t;

byte_t memory[NW_SIZE() * 2] = {0}; 

byte_t network[NW_SIZE()] = {
    0xDE, 0xAD, 0xAA,
#ifndef NO_CHECKSUM
    (0xDE ^ 0xAD ^ 0xAA),
#endif
    0xAD, 0xDA, 0xAD, 
#ifndef NO_CHECKSUM
    (0xAD ^ 0xDA ^ 0xAD),
#endif
    0xDE, 0xAD, 0xAA,
#ifndef NO_CHECKSUM
    (0xDE ^ 0xAD ^ 0xAA),
#endif
    0xAD, 0xDA, 0xAD,
#ifndef NO_CHECKSUM
    (0xAD ^ 0xDA ^ 0xAD),
#endif
#ifdef GLOBAL_CHECKSUM
#error "Not implemented"
#endif
    0xF5, 0xF5, 0xF5 
};

byte_t receiveData() {
    static unsigned nwstate = 0;

    if(nwstate >= (NW_SIZE()))
        exit(1);
    return network[nwstate++];
}

bool save_firmware(uint8_t* memory, firmware_t* firmware, unsigned adress)
{
    for(unsigned i = 0; i < PAGE_NUMBER; ++i) {
        page_t* page = &firmware->pages[i];
        uint8_t cs = computeChecksum(page);
        if(cs != page->checksum) {
            _LZ__CM("fcs");
        }
        
        unsigned offset = adress + (i * SIZEP);
        for(int k = 0; k < SIZEP; ++k) {
            memory[offset + k] = page->t[k]; 
        }
    }

    return true;
}

void init_firmware(firmware_t* firmware) {
    for(int i = 0; i < PAGE_NUMBER; ++i) {
        firmware->pages[i].checksum = 0;
        for(int k = 0; k < SIZEP; ++k)
            firmware->pages[i].t[k] = 0xFE;
    }
            
}

unsigned oracle_oob = 0;
unsigned oracle_diff = 0;

void verify_oracles(uint8_t* memory) {
    for(unsigned i = 0; i < PAGE_NUMBER; ++i) {
        unsigned i_fw = LOAD_ADDRESS + (i * SIZEP); // [B B B B B B]
        unsigned i_nw = (i * (SIZEP + 1)); // [B B B CS B B B CS]

        for(int k = 0; k < SIZEP; ++k) {
            if (memory[i_fw + k] != network[i_nw + k]) {
                oracle_diff++;
            }
        }
    }
} 

uint8_t computeChecksum(page_t* page_buffer) {
    // Fixed size.
    return page_buffer->t[0] ^ page_buffer->t[1] ^ page_buffer->t[2]; 
}

void firmware_updater() {
    firmware_t firmware;
    unsigned page_counter = 0;
    init_firmware(&firmware);

    // Lecture 
    int byte_counter = 0;
    while(true) {
        byte_t received = receiveData();
        page_t* page_buffer = &firmware.pages[page_counter];
        page_buffer->t[byte_counter] = received;
        byte_counter = byte_counter + 1;

        if(byte_counter >= SIZEP) { // End of page
            byte_t cs = receiveData(); // return expected checksum 
            // Compute checksum 
            page_buffer->checksum = computeChecksum(page_buffer);
            if(cs != page_buffer->checksum) {
                _LZ__CM("ics");
            }
            page_counter++;
            byte_counter = 0;
            
            if(page_counter >= PAGE_NUMBER) {
                break;
            }
            continue; // next page 
        }
    }


    // check that all pages have been (properly) transfered)
    if (page_counter != PAGE_NUMBER) {
        _LZ__CM("pc");
    }
    save_firmware(memory, &firmware, LOAD_ADDRESS);
}


bool oracle() {
    return (oracle_diff > 0) | (oracle_oob > 0);
}



bool valid() {
    for(unsigned i = 0; i < PAGE_NUMBER; ++i) {
        unsigned i_fw = LOAD_ADDRESS + (i * SIZEP); // [B B B B B B]
#ifndef NO_CHECKSUM
        unsigned i_nw = (i * (SIZEP + 1)); // [B B B CS B B B CS]
#else
        unsigned i_nw = i * SIZEP;
#endif
        for(int k = 0; k < SIZEP; ++k) {
            if (memory[i_fw + k] != network[i_nw + k]) {
                return false;
            }
        }
    }
    return true;
} 

bool oob() {
    for(unsigned i = 0; i < LOAD_ADDRESS; ++i) {
        if(memory[i] != 0) {
            return true;
        }
    }
    for(unsigned i = LOAD_ADDRESS + (SIZEP * PAGE_NUMBER); i < (NW_SIZE() * 2); ++i) {
        if(memory[i] != 0)
            return true;
    }
    return false;
}


int main()
{
    firmware_updater();

#ifdef LAZART
    _LZ__ORACLE(!valid() | oob());
#endif
}
\end{lstlisting} 


\chapter{Annexes supplémentaires}

    \section{Contraintes ILP pour le programme \texttt{memcmps3}}
    \label{annexe:ilp-memcmp}

        L'implémentation du placement optimal dans Lazart génère les contraintes d'un programme ILP dans le format de l'outil externe utilisé (voir section \ref{sec:placement-exps}). Le listing \ref{lst:ilp-memcmp-1} correspond aux contraintes pour le modèle de faute $DL$ en une faute et le listing \ref{lst:ilp-memcmp-3} corresponds aux contraintes générées pour le modèle de faute $TI+DL$ pour 3 fautes, le fichier de contraintes en quatre fautes comporte 136 lignes).

\lstset{caption={Contraintes \gls{ilp} pour le programme \texttt{memcmps3} ($DL$, 1 faute)},label=lst:ilp-memcmp-1,language=bash}
\begin{lstlisting}
# Ips
var ip7 >= 1;
var ip6 >= 1;
var ip14 >= 1;

minimize z: ip7 + ip6 + ip14;
subject to c0: ip14 >= 3;
subject to c1: ip6 + ip7 >= 3;
subject to c2: ip6 + ip14 >= 3;
subject to c3: ip6 + ip14 >= 3;
\end{lstlisting}

\lstset{caption={Contraintes \gls{ilp} pour le programme \texttt{memcmps3} ($TI+DL$, 3 fautes)},label=lst:ilp-memcmp-3,language=bash}
\begin{lstlisting}
# Ips
var ip12F >= 1;
var ip20 >= 1;
var ip13 >= 1;
var ip8 >= 1;
var ip9F >= 1;
var ip11 >= 1;
var ip10 >= 1;
var ip9T >= 1;

minimize z: ip12F + ip20 + ip13 + ip8 + ip9F + ip11 + ip10 + ip9T;
subject to c0: ip20 >= 4;
subject to c1: ip9F + ip20 >= 4;
subject to c2: ip8 + ip20 >= 4;
subject to c3: ip8 + ip10 >= 4;
subject to c4: ip9F + ip10 >= 4;
subject to c5: ip8 + ip20 >= 4;
subject to c6: ip8 + ip11 + ip20 >= 4;
subject to c7: ip9F + ip11 + ip20 >= 4;
subject to c8: ip9F + ip11 + ip20 >= 4;
subject to c9: ip8 + ip10 + ip20 >= 4;
subject to c10: ip8 + ip10 + ip11 >= 4;
subject to c11: ip9F + ip10 + ip11 >= 4;
subject to c12: ip8 + ip9T + ip20 >= 4;
subject to c13: ip8 + ip10 + ip11 >= 4;
subject to c14: ip8 + ip12F + ip20 >= 4;
subject to c15: ip8 + ip9F + ip10 >= 4;
subject to c16: ip8 + ip11 + ip20 >= 4;
subject to c17: ip8 + ip9F + ip20 >= 4;
subject to c18: ip9F + ip10 + ip12F >= 4;
subject to c19: ip9F + ip10 + ip11 >= 4;
subject to c20: ip9F + ip12F + ip20 >= 4;
subject to c21: ip8 + ip10 + ip12F >= 4;
subject to c22: ip8 + ip12F + ip13 >= 4;
subject to c23: ip9F + ip12F + ip13 >= 4;
subject to c24: ip9F + ip10 + ip20 >= 4;
subject to c25: ip9F + ip11 + ip13 >= 4;
subject to c26: ip8 + ip11 + ip13 >= 4;
\end{lstlisting}
